\chapter{Week Two}

% 24th October til 30th October

\paragraph{24. October, 7:45; Apartment B52}
It is raining.
The cold creeps in through the cracks.
We turn to the air conditioning for warmth.
Only time will tell if it helps.
Trying to boil water.
Just in case.
I hope the toaster works.

\paragraph{24. October, 9:50; Mount Gambier, Woolworths}
The Gadgeteer went to buy (another) umbrella.
The last did not survive the first time it was opened.

Started the day by visiting the sinkhole in the city center.
It looks like a giant toilet.
Consulting the plaques mounted around it, that is more or less what the city is using it for.

\paragraph{24. October, 11:11; Road to Adelaide}
Passing through a wine lover's wet dream.
Grape plants stretch from the road to the far horizon.
The regularity of the planting patterns mesmerizes the unsuspecting eye as we speed by.

And then, after a few kilometers, we are back to sheep.

\begin{center}
\begin{tabular}{||c||}
\emph{Achievement unlocked}\\
"Not a Very Nice Name For Retirement Homes"\\
\multicolumn{1}{||p{0.8\textwidth}||}{\footnotesize - Pass a Fossil Center} \\
\end{tabular}
\end{center}

\paragraph{24. October, 22:12; Adelaide Shore Resort, Apartment 30}
Went for dinner in Glennelg.
The bridge was out so we had to take a significantly less direct route.
Turns out even a famous nightlife spot will be more or less deserted on a Monday night.

Highlight was a short trip to the pier which sprung out pretty far into the dark sea and was occupied by a couple of fishing apparently Japanese guys-

\begin{center}
\begin{tabular}{||c||}
\emph{Achievement unlocked}\\
"Not A Banana Baton"\\
\multicolumn{1}{||p{0.8\textwidth}||}{\footnotesize - Have dinner in a palindrome town} \\
\end{tabular}
\end{center}

\paragraph{25. October, 10:35; Flight to Alice Springs}
The sky cleared just as we left.
Damn.
According to the forecast, we are going right into the damp cold rain in the center of bloody Red Center.
Not the kind of weather you associate with Australia.
Not the kind of weather you associate at all.
No more time to opt out.
The dies have been cast.
Here is hoping the bloody forecast was wrong.
At least we are on a domestic flight.
No bloody immigration controls and bloody lax security.

\begin{center}
\begin{tabular}{||c||}
\emph{Achievement unlocked}\\
"Contraband"\\
\multicolumn{1}{||p{0.8\textwidth}||}{\footnotesize - Bring containers of more than 100ml of any liquid onto a plane} \\
\end{tabular}
\end{center}

\paragraph{25. October, 15:28; Alice Springs}
So the weather forecast in Australia sucks.
We have arrived.
30�C - sunny.
We are roasting.
Using sun screen as marinade does not really help.
So we restocked on water.
And jerky.

Before heading on to Erldunda everyone donned an Australian leather hat to keep our brains from boiling over.
We also went to the Royal Flying Doctors of Australia museum.
Think REGA for Aussies.

Damn it is hot, the guy who invented air conditioning deserves a medal.

\paragraph{25. October, 15:40; Road to Erldunda}
We are driving on a straight road.
And I mean straight as in \emph{no bends whatsoever}.
We are passing signs that say "flood way"\ldots{}
There are depth indicators next to the road.
They go up to 2m.
I wonder if we should be concerned.

\paragraph{25. October, 22:25; Erldunda Station Bed and Breakfast}
We are now in the Outback.
Definitely.
No mobile phone reception.
The roads besides the main road are nothing but dirt.
And there are bushes, shrubbery and dirt as far as the eye can see.
And it is hot.
Sunscreen reserves are dwindling quickly.
People are getting headaches.
Bottled water is expensive.

The night sky is beautiful.
You do not realize how much light pollution there is back home.
Not until you have seen the milky way with bare eyes.
There certainly is a price tag attached to progress.
Maybe the reason people stopped reaching for the stars is because they no longer see them.

We have seen lightning and what looked like fire in the distance.
Maybe tomorrow we will know what that is all about.

\paragraph{26. October 14:15; King's Canyon}
I drove us a few hundred kilometers into nowhere to see King's Canyon.
We arrived in near roasting heat, filled up on water and hiked up the rusty red canyon walls.
Wind at our backs and sun on our hats we crossed the rocky tops.
We trailed along the towering cliffs, past birds and lizards, passing rock sculptures and fossilized sea grounds.

About halfway through the distance thunderstorm clouds rouse above the horizon.
They reached us just as we returned to the car park.
Yes, we got \emph{rained upon}.
In the \emph{desert}.
And there were still people up on the hike.
I suspect they got very wet.

\begin{center}
\begin{tabular}{||c||}
\emph{Achievement unlocked}\\
"Dry Heat My Ass"\\
\multicolumn{1}{||p{0.8\textwidth}||}{\footnotesize - Get drenched by rain in the Red Center} \\
\end{tabular}
\end{center}

\begin{center}
\begin{tabular}{||c||}
\emph{Achievement unlocked}\\
"Cleanliness is Next to Godliness"\\
\multicolumn{1}{||p{0.8\textwidth}||}{\footnotesize - Clean the windshields at a gas station while it is raining} \\
\end{tabular}
\end{center}

\paragraph{26. October 15:20; Road back to Erldunda}
It was just a quick shower.
We now see why there are signs warning about \emph{flood ways}-
There are large puddles spanning the width were there was road before and this after just a brief shower.
So the landlady at the station was not kidding.
When it rains, the flood meter \emph{really} go up to 2 meters.

\begin{center}
\begin{tabular}{||c||}
\emph{Achievement unlocked}\\
"Herd Crossing"\\
\multicolumn{1}{||p{0.8\textwidth}||}{\footnotesize - Stop for some wild horses crossing the road} \\
\end{tabular}
\end{center}

\paragraph{27. October 8:30; Road to Ayers Rock}
After a hearty breakfast at Erldunda Station we are back on the road.
The folk at the station are quite nice.
Dimension in this land are far off, the got about 7000 instances of cattle on about a million acres of land.
And they mostly manage them between the two of them.
At the moment the landlady's brother and parents are helping them out rounding up the cattle.
Somehow this is totally off scale.

On another notice we found out - that is - have been told what the little round fruity thingies growing alongside the road are.
Petty Melons.
Supposedly they taste \emph{disgusting}.

So now we left the station behind us, including the lizard hunting dog which drooled on my jeans yesterday.
We make our way to Uluru.

\paragraph{27. October 15:38; National Park}
We arrived at the park.
Once again in scalding heat.
Then we headed out for the scenic view point to see the Olgas, which sort of looked like a giant fake backdrop amidst the desert.
This is also where we meet our first \emph{Swiss} tourists.
Small world indeed.

Uluru - as well as the Olgas - is simply massive.
Both stand out in the flat desert from miles away.
And when you get closer perspective just grows over your head.
Even though they are dwarfed by most mountains and maybe a few hills back home.
Their solitary presence among the pool table that is Central Australia makes them appear monstrous.
Especially the mythical Uluru with his flanks scarred by erosion.
It is easy to imagine how the bizarre forms inspired the native people to their colorful folklore.
If you let your mind wander, you can imagine your own giant mythical creatures whose battles and corpses left their marks on the sandy rock.

\paragraph{27. October 18:45; Voyages Ayers Rock Resort, Room 123}
Continuing on the last train of thought - or rather - the last streak of reporting, we went for a hike though the Olgas.
Halfway to the second lookout, the signs were once again pointing towards rain.
Yes, rain.
In the desert.
\emph{Again}.
So the team split up into people wanting to not get wet and people who did not mind hiking in the rain.
The Gadgeteer, the Bomber and me went on, trailing the Valley of Wind deeper into the cliffs.
We climbed a few stone slopes, bested rickety steps and ducked under low vegetation to the second outlook.
We were greeted by a great view and the rain.

First thing we noticed: The hats do not only do a great job against scalding sun, they are also quite waterproof.
Second thing we noticed: The path we trailed back on doubles as a natural drainpipe for both the cliff faces we found ourselves in between.

What was before an arid sculpture of rock was transformed.
The rain turned it into a damp fountain of streams and waterfalls
We made our way back safely, across slippery stones and gurgling creeks.
And we encountered some souls that were even more adventurous as they were hiking \emph{up} the path during the rain.
We reached the car park even before the do-not-get-wets came back from their excursion.

I am now writing this in an upgraded apartment.
We should be put in bunk beds, but capacity limits seem to have granted us a free upgrade.
Yay.
Part of that seems to be because of some boss-has-brother-in-Switzerland bonus.
Hooray for being Swiss.

Next thing I have been advised to log is the horribly inaccurate fuel gauge of our current rental car.
We have been driving on a tank reported empty for about an hour.

\begin{center}
\begin{tabular}{||c||}
\emph{Achievement unlocked}\\
"Swissness\textsuperscript{TM}"\\
\multicolumn{1}{||p{0.8\textwidth}||}{\footnotesize - Get perks because you are Swiss} \\
\end{tabular}
\end{center}

\begin{center}
\begin{tabular}{||c||}
\emph{Achievement unlocked}\\
"Thunderstruck"\\
\multicolumn{1}{||p{0.8\textwidth}||}{\footnotesize - Get into a thunderstorm in the desert} \\
\end{tabular}
\end{center}

\begin{center}
\begin{tabular}{||c||}
\emph{Achievement unlocked}\\
"Blind Gauge"\\
\multicolumn{1}{||p{0.8\textwidth}||}{\footnotesize - Drive on a broken fuel gauge pointing to empty} \\
\end{tabular}
\end{center}

\begin{center}
\begin{tabular}{||c||}
\emph{Achievement unlocked}\\
"Ayers Falls"\\
\multicolumn{1}{||p{0.8\textwidth}||}{\footnotesize - Visit Uluru while it is raining} \\
\end{tabular}
\end{center}

\paragraph{27. October 20:40; Voyages Ayers Rock Resort, Room 123}
Free Interwebs was a lie, we had to buy our way into the wireless system.
Also, four bottles of cider border on the problematic.
Never start drinking with a traveling lone Korean woman.
The one we encountered seems atypically drinking-proofed and had a lot of games, tricks and whatnot ready to get us drunk and paying for her drinks.
I managed to pull the plug before it got critical, but Red October is still out there\ldots{}

\paragraph{28. October 7:30; Road to Uluru}
Stood up early to see the sun rise over Uluru.
It was still raining.
Uluru was being a diva.
The sight of rain soaked Uluru covered in clouds is supposed to be rare.
When the rain clouds are hiding the sun, sunrise is still not that impressive.

\paragraph{28. October 10:10; Road to Yulara}
Joined a ranger guided walk around the base of Uluru.
We did not walk far.
The guide's explanation shed some understanding on the way of the Anangu.
And it shed some light on the reasons they have for their set of rules.

\paragraph{28. October 14:25; Connellan Airport}
Continuing from the previous thought while plane engines are screeching in the background.
The stories and myths are their way of preserving the knowledge vital to survival.
Rules, landmarks, ways to travel the heat, craft.
All those things are passed to the next generation by stories.
And the landscape serves as a remainder to tie those stories to.
A natural encyclopedia that you need to learn how to read.
And their way of belief always ties a story to a place.
As the knowledge about something is meaningless if that thing is not present, so it is only told where it still holds meaning.
A photograph on those terms holds no value.
It does not contain anything that helps you somewhere else.
And it might be dangerous, applying knowledge of one place to another, so it is kept to where it works, where it has meaning.
It is hard to understand a culture if you bring your own.

We did not climb Uluru, partially because it was closed off, partially because there is a simple explanation why you should not.
Uluru is a natural drinking water storage.
You would not go swim in your drinking water.
Much less walk over it with sweaty clothes and dirty shoes\ldots{}

\begin{center}
\begin{tabular}{||c||}
\emph{Achievement unlocked}\\
"Tjukurpa"\\
\multicolumn{1}{||p{0.8\textwidth}||}{\footnotesize - Do not climb Uluru} \\
\end{tabular}
\end{center}

\paragraph{28. October 22:23; Pacific Inn, Room 916}
Arrived at Cairns, ready for the final drive to Sydney.
We checked out the same car as we had in the Red Center, a KIA Grand Carnival\ldots{}
The circus is back in town.

Went out for free Internet WiFi at McDonalds.
Poor quality access, I guess there are a lot of freeloaders and little bandwidth.

So we returned to the hotel and bought a 24h access package.
Beforehand The Gadgeteer and The Bomber set up a wireless router as NAT and registered that device with the hotel.
We are accessing the net with all out devices using the same connection package.
Engineers ho!

\begin{center}
\begin{tabular}{||c||}
\emph{Achievement unlocked}\\
"Ad Hoc Commissioning"\\
\multicolumn{1}{||p{0.8\textwidth}||}{\footnotesize - Set up your own wireless hot spot using the hotel's wired access} \\
\end{tabular}
\end{center}

\paragraph{29. October 14:25; Kuranda, Village Vibe Caf�}
At this moment, we find ourselves slurping Iced Chai in a tourist trap amidst some rainforesty Jungle north of Cairns.
We came here by rope way which was constructed by the French as we noted horrified halfway up the mountain.

Then we strolled through the rain forest on concrete paths.

\paragraph{29. October 15:50; Scenic Railroad}
Rolling along a railway track which is about a century old.
We get treated to a view of Barron Falls and Robb's Monument, from which at one point "the Italian flag was flown, to show their courage".
Well, during the train ride the Australian audio commentary was being stuttered, showing their engineering skills.

The way down is mostly accompanied by the creaking of wooden car chassis, the smell of diesel fumes and the rattling of wheels against rusty track.
And we are passing sooty tunnels, rickety bridge and deep gorges.

Quick update on toilet paper status:
Still unused but it has already sustained some minor damage.

\paragraph{29. October 16:45; Still on the Railroad}
While I was writing postcards, the broken audio record continued narrating and skipping over part of the narration.
From what I was able to make out, building the railroad was difficult.
They used methodist churches, hotels and had to bring their own bottles or was that tools?
Also there seemed to be a governor which was invited to a banquet on a bridge but could not hold a speech because of the waterfall.
Also there were hotels on the edge and families which were having a hard\ldots \texttt{<record ends here>}
And there were snakebites and accidents.
But maybe I am messing things up a little.

Still, though piece of work this stretch of railroad.

\paragraph{30. October 9:00; Parking Lot, Pacific International Hotel}
Still feeling full of BBQ dinner from yesterday.
We have had only a light breakfast and are now waiting for The Preceptress and The Gadgeteer to complete the checkout.
The front desk does not seem very efficient.

Ok, it was just trouble with The Preceptress' credit card.
She and credit cards do not seem to go together.

Now the long drive beckons.
Airly Beach is the goal, a ship cruise awaits.

Let us see whether I get seasick or not.
